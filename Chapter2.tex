\chapter{Les tests logiciels et leurs automatisations}
\thispagestyle{fancy}

\section{L'interêt des tests logiciels}
\label{sec:interet}
Les tests logiciels sont un processus systématique de vérification du logiciel pour s'assurer de son fonctionnement correct et de la conformité aux exigences spécifiées.
Les tests logiciels sont importants pour garantir la qualité et la fiabilité du produit logiciel. Ils permettent de détecter les erreurs et les bugs avant la livraison, ce qui peut réduire les coûts liés à la correction des erreurs en production. Les tests logiciels peuvent également renforcer la confiance des clients et des utilisateurs finaux dans la qualité du produit, ce qui peut améliorer la réputation de l'entreprise et augmenter les chances de vente. Bien que les tests logiciels nécessitent un investissement en temps et en ressources, ils peuvent finalement permettre de réaliser des économies durant toute la durée de vie des produits \parencite{interet}.

% section with reference to bibliography
\section[Classification des tests]{Classification des tests logiciels par \parencite{classification}}
\label{sec:classification}
Les tests logiciels peuvent être classifiés en fonction de différents critères tels que la catégorie, le niveau, la technique et les types. Chacun de ces critères a ses propres avantages et inconvénients.
%initiate list environment
\begin{itemize}
\item Catégorie :
Les tests peuvent être classés en tests dynamiques et tests statiques. Les tests dynamiques peuvent être automatisés, tandis que les tests statiques sont généralement effectués manuellement. Les tests dynamiques ont l'avantage d'être plus rapides et plus fiables, mais peuvent être coûteux à mettre en place. Les tests statiques, quant à eux, sont moins coûteux, mais nécessitent une intervention humaine pour les effectuer et peuvent être plus longs et moins fiables.

\item Niveau :
Les tests peuvent également être classés en fonction du niveau des composants téstées. Les tests de niveau unitaire sont effectués sur des composants individuels du logiciel comme par exemple des méthodes et des fonctions.Un peu plus haut tests d'intégration sont effectués lorsque plusieurs composants du logiciel sont combinés et testés ensemble. Les tests de système sont effectués sur le système dans son intégralité. Enfin, les tests d'acceptation sont effectués pour vérifier si le produit répond aux exigences des utilisateurs. Chacun de ces niveaux de test a des avantages et des inconvénients spécifiques en termes de coût, de rapidité et de fiabilité.

\item Technique :
Lorsqu'il s'agit de la technique de test, on distingue principalement les tests non fonctionnels, également appelés tests en boîte blanche, et les tests fonctionnels, appelés tests en boîte noire. La différence qui distingue les tests en boîte noire (BBT) des tests en boîte blanche (WBT) est la question de "perspective". Le BBT considère les objets à tester comme une boîte noire en se concentrant sur les relations entrée-sortie ou sur le comportement fonctionnel externe. Tandis que le WBT considère les objets comme une boîte transcparente où les détails d'implémentation interne sont visibles et testés. Le BBT et le WBT peuvent également être comparés en fonction de la façon dont ils abordent les questions suivantes : objets, chronologie, focus sur les défauts, détection et correction des défauts. Le BBT est généralement effectué par des testeurs professionnels dédiés et peut également être effectué par des tiers dans un contexte vérification et validation indépendantes, tandis que le WBT est souvent effectué par les développeurs eux-mêmes \parencite{SoftwareQualityEngineering}.

\item Types : Enfin, il existe des tests spécifique comme les tests de performance, de sécurité ou d'utilisabilité. Ils sont effectué par des experts dans ces domaines \parencite{classification}.



\end{itemize}



\section[les tests d'acceptation et l'automatisation]{L'interêt des tests d'acceptation et l'importance de leur automatisation}
Les tests d'acceptation visent à évaluer le logiciel du point de vue de l'utilisateur et se concentrent sur les scénarios d'utilisation, les séquences, les modèles et les fréquences associées. Ils sont généralement plus adaptés aux logiciels lourds et leurs composants en tant qu'ensemble. Ils ont une relation étroite avec les clients et les utilisateurs et sont effectués dans les phases finales de test. L'environnement de test est similaire à celui utilisé par l'utilisateur final. 



\label{sec:criteres}
\begin{center}
    

\begin{tabular}{|c|c|}
    
    \hline
    \hline
    & Tests d'acceptation  \\ \hline
    Catégorie  & statique ou dynamique \\ \hline
    Technique & boite noir \\ \hline
    Responsable & les testeurs et les developpeurs \\ \hline
    Quand ? & après le développement \\ \hline
    Pourquoi ? & pour vérifier si le produit répond aux exigences des utilisateurs \\ \hline
    \hline
\end{tabular}
\end{center}
L'automatisation implique le développement de scripts de test à l'aide de langages de programmation tels que Python, JavaScript ou Java afin que les cas de test puissent être exécutés automatiquement avec un minimum d'intervention humaine. Ceci a l'avantage de réduire l'effort humain et économiser de l'argent. Le logiciel d'automatisation peut également saisir des données de test dans le système, comparer les résultats attendus et réels et aussi générer des rapports de test détaillés \parencite{automation}. Le test automatisé nécessite des investissements considérables en argent et en ressources. Il est également possible d'enregistrer les suites de tests et de les rejouer au besoin. Une fois la suite de tests automatisée, aucune intervention humaine n'est nécessaire. Le but de l'automatisation est de réduire le nombre de cas de test à exécuter manuellement et non pas d'éliminer complètement les tests manuels. 
Les avantages des tests automatisés sont nombreux: la rapidité, l'efficacité, la répétabilité, la réutilisabilité, la programmabilité, la couverture complète et la fiabilité. Malgré tout ces avantages il n'est pas toujours nécessaire, approprié ou rentable d'automatiser ces tests, mais une analyse peut aider à déterminer les bénéfices d'un test automatisé par rapport à un test manuel. La bonne gestion peut être obtenue en identifiant et en estimant les coûts et les avantages du test automatisé \parencite{automation2}.

