

\chapter{Introduction}

\label{chap:intro}

Les tests d'acceptation sont un élément clé de la qualité des produits logiciels. Ils permettent de vérifier que le produit répond aux exigences du client et peut être utilisé de manière efficace. Dans un environnement agile, où la livraison rapide et la flexibilité sont des priorités, il est important de mettre en place une approche efficace pour les tests d'acceptation.

L'objectif de cet article bibliographique est d'explorer les différents aspects de la mise en place des tests d'acceptation dans un environnement agile. Nous examinerons les avantages et les défis de l'intégration des tests d'acceptation dans une méthodologie agile, ainsi que les meilleures pratiques pour les mettre en œuvre avec succès. Nous explorerons également les outils et les technologies disponibles pour soutenir les tests d'acceptation agiles.

Cet article bibliographique aidera les professionnels du développement logiciel qu'ils soient programmeur,testeurs ou experts métiers qui cherchent à comprendre les tests et leurs raport avec les méthodes agiles.