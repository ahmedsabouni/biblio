\chapter{Introduction}
\label{chap:intro}


L'automatisation des tests prend une importance croissante dans le développement des logiciels. Elle garantit la qualité et la fiabilité des applications en réduisant le risque de bugs et en améliorant les performances. L'automatisation des tests permet également de réduire les coûts et les délais de développement en automatisant les tâches répétitives et en permettant aux équipes de se concentrer sur des tâches plus complexes. Ce sujet est particulièrement pertinent dans le contexte du développement agile, où les exigences et les priorités peuvent changer rapidement.


Les tests de régression sont un aspect clé de l'automatisation des tests. Ils permettent de vérifier que les modifications apportées à une application ne causent pas d'erreurs ou de bugs dans les fonctionnalités existantes. Les tests de régression sont particulièrement importants dans les environnements de développement où les modifications sont fréquentes et où les délais de développement sont courts. L'automatisation des tests de régression permet de réduire considérablement les coûts et les délais liés à cette tâche, tout en augmentant la fiabilité et la qualité des applications. Les tests de régression automatisés peuvent également être exécutés de manière régulière, ce qui permet de détecter les erreurs dès leur apparition.

Dans cet article, nous allons explorer les avantages de l'automatisation des tests de régression dans un environnement agile, les différentes techniques et outils utilisés pour automatiser ces tests, et les meilleures pratiques pour mettre en place une stratégie efficace d'automatisation des tests de régression. Nous allons également discuter des défis courants auxquels les équipes peuvent être confrontées lors de la mise en place d'une stratégie d'automatisation des tests de régression et comment les surmonter.
\parencite{classification}